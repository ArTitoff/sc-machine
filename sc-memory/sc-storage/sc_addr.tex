\scnheader{sc-адрес}
\scnidtf{адрес элемента sc-хранилища, соответствующего заданному sc-элементу, в рамках текущего состояния реализации sc-хранилища в составе программной модели sc-памяти}
\scntext{пояснение}{Каждый элемент sc-хранилища в текущей реализации может быть однозначно задан его адресом (sc-адресом), состоящим из номера сегмента и номера \textit{элемента sc-хранилища} в рамках сегмента. Таким образом, \textit{sc-адрес} служит уникальными координатами \textit{элемента sc-хранилища} в рамках \textit{Реализации sc-хранилища}.}
\scntext{примечание}{Sc-адрес никак не учитывается при обработке базы знаний на семантическом уровне и необходим только для обеспечения доступа к соответствующей структуре данных, хранящейся в линейной памяти на уровне \textit{Реализации sc-хранилища}.}
\scntext{примечание}{В общем случае sc-адрес элемента sc-хранилища, соответствующего заданному sc-элементу, может меняться, например, при пересборке базы знаний из исходных текстов и последующем перезапуске системы. При этом sc-адрес элемента sc-хранилища, соответствующего заданному sc-элементу, непосредственно в процессе работы системы в текущей реализации меняться не может.}
\scntext{примечание}{Для простоты будем говорить "sc-адрес sc-элемента"{}, имея в виду \textit{sc-адрес} \textit{элемента sc-хранилища}, однозначно соответствующего данному \textit{sc-элементу}.}
\begin{scnrelfromlist}{семейство отношений, однозначно задающих структуру заданной сущности}
    \scnitem{номер сегмента sc-хранилища*}
    \scnitem{номер элемента sc-хранилища в рамках сегмента*}
\end{scnrelfromlist}
\scntext{примечание}{Для каждого sc-адреса можно взаимно однозначно поставить в соответствие некоторый хэш, полученный в результате применения специальной хэш-функции над этим sc-адресом. Хэш является неотрицательным целым числом и является результатом преобразования номера сегмента sc-хранилища si, в котором располагается sc-элемент, и номера этого sc-элемента sc-хранилища ei в рамках этого сегмента si. В рамках sc-хранилища используется единственная хеш-функция для получения хеша sc-адреса sc-элемента и задаётся как $f(si, ei) = si << 16 \vee ei \wedge 0xffff$, где операция $<<$ - операция логического битового сдвига влево левого аргумента на количество единиц, заданное правым аргументом, относительно этой операции, операция $\vee$ - операция логического ИЛИ, операция $\wedge$ - операция логического И, число $0xffff$ - число 65535, представленное в шестнадцатеричном виде и обозначающее максимальное количество sc-элементов в одном сегменте sc-хранилища.}
