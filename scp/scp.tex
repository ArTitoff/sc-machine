\scnheader{Реализация scp-интерпретатора}
\scnrelto{программная реализация}{Абстрактная scp-машина}
\scntext{примечание}{Важнейшей особенностью Языка SCP является тот факт, что его программы записываются таким же образом, что и обрабатываемые ими знания, то есть в SC-коде. Это, с одной стороны, дает возможность сделать ostis-системы платформенно-независимыми (четко разделить \textit{sc-модель компьютерной системы} и платформу интерпретации таких моделей), а с другой стороны требует наличия в рамках платформы \textit{Реализации scp-интерпретатора}, то есть интерпретатора программ Языка SCP.}
\begin{scnrelfromlist}{используемый язык программирования}
    \scnitem{C++}
\end{scnrelfromlist}
\begin{scnrelfromlist}{компонент программной системы}
    \scnitem{Реализация Абстрактного sc-агента создания scp-процессов;Реализация Абстрактного sc-агента интерпретации scp-операторов}
    \begin{scnindent}
        \begin{scnrelfromlist}{компонент программной системы}
            \scnitem{Реализация Абстрактного sc-агента интерпретации scp-операторов генерации конструкций}
            \scnitem{Реализация Абстрактного sc-агента интерпретации scp-операторов ассоциативного поиска конструкций}
            \scnitem{Реализация Абстрактного sc-агента интерпретации scp-операторов удаления конструкций}
            \scnitem{Реализация Абстрактного sc-агента интерпретации scp-операторов проверки условий}
            \scnitem{Реализация Абстрактного sc-агента интерпретации scp-операторов управления значениями операндов}
            \scnitem{Реализация Абстрактного sc-агента интерпретации scp-операторов управления scp-процессами}
            \scnitem{Реализация Абстрактного sc-агента интерпретации scp-операторов управления событиями}
            \scnitem{Реализация Абстрактного sc-агента интерпретации scp-операторов обработки содержимых числовых файлов}
            \scnitem{Реализация Абстрактного sc-агента интерпретации scp-операторов обработки содержимых строковых файлов}
        \end{scnrelfromlist}
    \end{scnindent}
    \scnitem{Реализация Абстрактного sc-агента синхронизации процесса интерпретации scp-программ}
    \scnitem{Реализация Абстрактного sc-агента уничтожения scp-процессов}
    \scnitem{Реализация Абстрактного sc-агента синхронизации событий в sc-памяти и ее реализации}
\end{scnrelfromlist}
\scntext{примечание}{Текущая \textit{Реализация scp-интерпретатора} не включает в себя специализированных средств для работы с блокировками, поскольку механизм блокировок элементов sc-памяти реализован на более низком уровне в рамках \textit{Реализация sc-хранилища и механизма доступа к нему}}
