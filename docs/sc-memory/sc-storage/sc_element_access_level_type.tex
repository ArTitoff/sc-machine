\scnheader{метка уровня доступа sc-элемента}
\scnrelto{второй домен}{метка уровня доступа sc-элемента*}
\begin{scnrelfromset}{обобщенная структура}
    \scnitem{метка уровня доступа sc-элемента на чтение}
    \scnitem{метка уровня доступа sc-элемента на запись}
\end{scnrelfromset}
\scntext{пояснение}{В текущей \textit{Реализации sc-хранилища} \textit{метки уровня доступа} используются для того, чтобы обеспечить возможность ограничения доутспа некоторых процессов в sc-памяти к некоторым sc-элементам, хранимым в sc-памяти.

Каждому элементу sc-хранилища соответствует \textit{метка уровня доступа sc-элемента на чтение} и \textit{метка уровня доступа sc-элемента на запись}, каждая из которых выражается числом от 0 до 255.

В свою очередь, каждому процессу (чаще всего, соответствующему некоторому sc-агенту), который пытается получить доступ к данному элементу sc-хранилища (прочитать или изменить его) соответствует уровень доступа на чтение и запись, выраженный в том же числовом диапазоне. Указанный уровень доступа для процесса является частью \textit{контекста процесса}. Доступ на чтение или запись к элементу sc-хранилища не разрешается, если уровень доступа соответственно на чтение или запись у процесса ниже, чем у элемента sc-хранилища, к которому осуществляется доступ.

Таким образом нулевое значение \textit{метки уровня доступа sc-элемента на чтение} и \textit{метки уровня доступа sc-элемента на запись} означает, что любой процесс может получить неограниченный доступ к данному элементу sc-хранилища.}
