\scnheader{метка синтаксического типа sc-элемента}
\scnidtf{уникальный числовой идентификатор, однозначно соответствующий заданному типу sc-элементов и приписываемый соответствующему элементу sc-хранилища на уровне реализации}
\scntext{примечание}{Очевидно, что тип (класс, вид) sc-элемента в sc-памяти может быть задан путем явного указания принадлежности данного sc-элемента соответствующему классу (sc-узел, sc-дуга и т.д.).

Однако, в рамках \textit{платформы интерпретации sc-моделей компьютерных систем} должен существовать какой-либо набор \textit{меток синтаксического типа sc-элемента}, которые задают тип элемента на уровне платформы и не имеют соответствующей sc-дуги принадлежности (а точнее -- базовой sc-дуги), явно хранимой в рамках sc-памяти (ее наличие подразумевается, однако она не хранится явно, поскольку это приведет к бесконечному увеличению числа sc-элементов, которые необходимо хранить в sc-памяти). Как минимум, должна существовать метка, соответствующая классу \textit{базовая sc-дуга}, поскольку явное указание принадлежности sc-дуги данному классу порождает еще одну \textit{базовую sc-дугу}.

Таким образом, \textit{базовые sc-дуги}, обозначающие принадлежность sc-элементов некоторому известному ограниченному набору классов представлены неявно. Этот факт необходимо учитывать в ряде случаев, например, при проверке принадлежности sc-элемента некоторому классу, при поиске всех выходящих sc-дуг из заданного sc-элемента и т.д.

При необходимости некоторые из таких неявно хранимых sc-дуг могут быть представлены явно, например, в случае, когда такую sc-дугу необходимо включить в какое-либо множество, то есть провести в нее другую sc-дугу. В этом случае возникает необходимость синхронизации изменений, связанных с данной sc-дугой (например, ее удалении), в явном и неявном ее представлении. В текущей \textit{Реализации sc-хранилища} данный механизм не реализован.

Таким образом, полностью отказаться от \textit{меток синтаксического типа sc-элементов} невозможно, однако увеличение их числа хоть и повышает производительность платформы за счет упрощений некоторых операций по проверке типов sc-элемента, но приводит к увеличению числа ситуаций, в которых необходимо учитывать явное и неявное представление sc-дуг, что, в свою очередь, усложняет развитие платформы и разработку программного кода для обработки хранимых sc-конструкций.}
\scnrelto{второй домен}{метка синтаксического типа sc-элемента*}
\scnsuperset{метка sc-узла}
\begin{scnindent}
\scntext{числовое выражение в шестнадцатеричной системе}{0x1}
\end{scnindent}
\scnsuperset{метка внутреннего файла ostis-системы}
\begin{scnindent}
\scntext{числовое выражение в шестнадцатеричной системе}{0x2}
\end{scnindent}
\scnsuperset{метка sc-ребра общего вида}
\begin{scnindent}
\scntext{числовое выражение в шестнадцатеричной системе}{0x4}
\end{scnindent}
\scnsuperset{метка sc-дуги общего вида}
\begin{scnindent}
\scntext{числовое выражение в шестнадцатеричной системе}{0x8}
\end{scnindent}
\scnsuperset{метка sc-дуги принадлежности}
\begin{scnindent}
\scntext{числовое выражение в шестнадцатеричной системе}{0x10}
\end{scnindent}
\scnsuperset{метка sc-константы}
\begin{scnindent}
\scntext{числовое выражение в шестнадцатеричной системе}{0x20}
\end{scnindent}
\scnsuperset{метка sc-переменной}
\begin{scnindent}
\scntext{числовое выражение в шестнадцатеричной системе}{0x40}
\end{scnindent}
\scnsuperset{метка позитивной sc-дуги принадлежности}
\begin{scnindent}
\scntext{числовое выражение в шестнадцатеричной системе}{0x80}
\end{scnindent}
\scnsuperset{метка негативной sc-дуги принадлежности}
\begin{scnindent}
\scntext{числовое выражение в шестнадцатеричной системе}{0x100}
\end{scnindent}
\scnsuperset{метка нечеткой sc-дуги принадлежности}
\begin{scnindent}
\scntext{числовое выражение в шестнадцатеричной системе}{0x200}
\end{scnindent}
\scnsuperset{метка постоянной sc-дуги}
\begin{scnindent}
\scntext{числовое выражение в шестнадцатеричной системе}{0x400}
\end{scnindent}
\scnsuperset{метка временной sc-дуги}
\begin{scnindent}
\scntext{числовое выражение в шестнадцатеричной системе}{0x800}
\end{scnindent}
\scnsuperset{метка небинарной sc-связки}
\begin{scnindent}
\scntext{числовое выражение в шестнадцатеричной системе}{0x80}
\end{scnindent}
\scnsuperset{метка sc-структуры}
\begin{scnindent}
\scntext{числовое выражение в шестнадцатеричной системе}{0x100}
\end{scnindent}
\scnsuperset{метка ролевого отношения}
\begin{scnindent}
\scntext{числовое выражение в шестнадцатеричной системе}{0x200}
\end{scnindent}
\scnsuperset{метка неролевого отношения}
\begin{scnindent}
\scntext{числовое выражение в шестнадцатеричной системе}{0x400}
\end{scnindent}
\scnsuperset{метка sc-класса}
\begin{scnindent}
\scntext{числовое выражение в шестнадцатеричной системе}{0x800}
\end{scnindent}
\scnsuperset{метка абстрактной сущности}
\begin{scnindent}
\scntext{числовое выражение в шестнадцатеричной системе}{0x1000}
\end{scnindent}
\scnsuperset{метка материальной сущности}
\begin{scnindent}
\scntext{числовое выражение в шестнадцатеричной системе}{0x2000}
\end{scnindent}
\scnsuperset{метка константной позитивной постоянной sc-дуги принадлежности}
\begin{scnindent}
\scnidtf{метка базовой sc-дуги}
\scnidtf{метка sc-дуги основного вида}
\begin{scnreltoset}{пересечение}
    \scnitem{метка sc-дуги принадлежности}
    \scnitem{метка sc-константы;метка позитивной sc-дуги принадлежности}
    \scnitem{метка постоянной sc-дуги}
\end{scnreltoset}
\scntext{примечание}{\textit{метки синтаксических типов sc-элементов} могут комбинироваться между собой для получения более частных классов меток. С точки зрения программной реализации такая комбинация выражается операцией побитового сложения значений соответствующих меток.}
\end{scnindent}
\scnsuperset{метка переменной позитивной постоянной sc-дуги принадлежности}
\begin{scnindent}
\begin{scnreltoset}{пересечение}
    \scnitem{метка sc-дуги принадлежности}
    \scnitem{метка sc-переменной}
    \scnitem{метка позитивной sc-дуги принадлежности;метка постоянной sc-дуги}
\end{scnreltoset}
\end{scnindent}
\scntext{примечание}{Числовые выражения некоторых классов меток могут совпадать. Это сделано для уменьшения размера элемента sc-хранилища за счет уменьшения максимального размера метки. Конфликт в данном случае не возникает, поскольку такие классы меток не могут комбинироваться, например \textit{метка ролевого отношения} и \textit{метка нечеткой sc-дуги принадлежности}.}
\scntext{примечание}{Важно отметить, что каждому из выделенных классов меток (кроме классов, получаемых путем комбинации других классов) однозначно соответствует порядковый номер бита в линейной памяти, что можно заметить, глядя на соответствующие числовые выражения классов меток. Это означает, что классы меток не включаются друг в друга, например, указание \textit{метки позитивной sc-дуги принадлежности} не означает автоматическое указание \textit{метки sc-дуги принадлежности}. Это позволяет сделать операции комбинирования и сравнения меток более эффективными.}
\begin{scnreltoset}{недостатки текущего состояния}
    \scnitem{\scnfilelong{На данный момент число \textit{меток синтаксического типа sc-элемента} достаточно велико, что приводит к возникновению достаточно большого числа ситуаций, в которых нужно учитывать явное и неявное хранение sc-дуг принадлежности соответствующим классам. С другой стороны, изменение набора меток с какой-либо целью в текущем варианте реализации представляет собой достаточно трудоемкую задачу (с точки зрения объема изменений в программном коде платформы и sc-агентов, реализованных на уровне платформы), а расширение набора меток без увеличения объема элемента sc-хранилища в байтах оказывается и вовсе невозможным.}}
    \begin{scnindent}
        \scntext{вариант решения}{Решением данной проблемы является максимально возможная минимизация числа меток, например, до числа меток, соответствующих \textit{Алфавиту SC-кода}. В таком случае принадлежность sc-элементов любым другим классам будет записываться явно, а число ситуаций, в которых необходимо будет учитывать неявное хранение sc-дуг, будет минимальным.}
    \end{scnindent}
    \scnitem{\scnfilelong{Некоторые метки из текущего набора \textit{меток синтаксического типа sc-элемента} используются достаточно редко (например, \textit{метка sc-ребра общего вида} или \textit{метка негативной sc-дуги принадлежности}), в свою очередь, в sc-памяти могут существовать классы, имеющие достаточно много элементов (например, \textit{бинарное отношение} или \textit{число}). Данный факт не позволяет в полной мере использовать эффективность наличия меток.}}
    \begin{scnindent}
        \scntext{вариант решения}{Решением данной проблемы является отказ от заранее известного набора меток и переход к динамическому набору меток (при этом их число может оставаться фиксированным). В этом случае набор классов, выражаемых в виде меток будет формироваться на основании каких-либо критериев, например, числа элементов данного класса или частоты обращений к нему.}
    \end{scnindent}
\end{scnreltoset}
