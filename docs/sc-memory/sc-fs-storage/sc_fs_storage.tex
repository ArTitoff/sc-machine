\scnheader{Реализация файловой памяти ostis-системы}
\scntext{пояснение}{Для хранения содержимого внутренних файлов ostis-систем, размер которого превышает 48 байт, используются файлы, явно хранимые на файловой системе, доступ к которой осуществляется средствами операционной системы, на которой работает \textit{Программный вариант реализации платформы интерпретации sc-моделей компьютерных систем}.

В общем случае множество различных внутренних файлов ostis-системы могут иметь одинаковое содержимое. Было бы разумно не хранить содержимое одинаковых файлов дважды. Для этого при создании соответствуюещго sc-узла и указании файла на файловой системе, который является содержимым данного sc-узла, вычисляется hash-сумма содержимого с помощью алгоритма SHA256. В результате получается строка из 32 символов, которая и выступает в качестве \textit{содержимого элемента sc-хранилища*}. Само же содержимое копируется в
файл на файловой системе, путь к которому строится на основании hash-суммы. Рядом с этим файлом создается файл, в котором хранятся sc-адреса всех sc-узлов, имеющих одно и то же ранее указанное содержимое. Таким образом, для того, чтобы найти все sc-узлы, имеющие указанное содержимое, необходимо вычислить hash-сумму искомого содержимого-образца и проверить наличие файла на файловой системе по пути, вычисляемому из hash-суммы и если он существует, то вернуть список хранящихся sc-адресов.

Кроме того, для реализации быстрого поиска sc-элементов по их строковым sc-идентификаторам или их фрагментам (подстрокам) используется дополнительное хранилище вида ключ-значение, которое ставит в соответствие \textit{строковому sc-идентификатору} \textit{sc-адрес} того \textit{sc-элемента}, идентификатором которого является данная строка (в случае основного и системного sc-идентификатора) или \textit{sc-элемента}, который является знаком \textit{внутреннего файла ostis-системы} (в случае неосновного sc-идентификатора).}
