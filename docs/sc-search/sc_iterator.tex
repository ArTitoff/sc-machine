\scnheader{sc-итератор}
\scnidtf{ScIterator}
\scnrelto{класс компонентов}{Реализация sc-хранилища}
\scnexplanation{С функциональной точки зрения \textit{sc-итераторы} как часть \textit{Реализации sc-хранилища} представляют собой базовое средство доступа к конструкциям, хранимым в sc-памяти, которое позволяет осуществить чтение (просмотр) конструкций, изоморфных простейшим шаблонам -- \textit{трехэлементным sc-конструкциям} и \textit{пятиэлементным sc-конструкциям}.

С точки зрения реализации \textit{sc-итератор} представляет собой структуру данных, которая соответствует определенному дополнительно уточняемому классу sc-конструкций и позволяет при помощи соответствующего набора функций последовательно осуществлять просмотр всех sc-конструкций данного класса, представленных в текущем состоянии sc-памяти (итерацию по sc-конструкциям).

Каждому классу \textit{sc-итераторов} соответствует некоторый известный класс (шаблон, образец) \mbox{sc-конструкций}. При создании sc-итератора данный шаблон уточняется, то есть некоторым (как минимум одному) элементам шаблона ставится в соответствие конкретный заранее известный \textit{sc-элемент} (отправная точка при поиске), а другим элементам шаблона (тем, которые нужно найти) ставится в соответствие некоторый тип sc-элемента из числа типов, соответствующих \textit{меткам синтаксического типа sc-элемента}.

Далее путем вызова соответствующей функции (или метода класса в ООП) осуществляется последовательный просмотр всех sc-конструкций, соответствующих полученному шаблону (с учетом указанных типов sc-элементов и заранее заданных известных sc-элементов), то есть \textit{sc-итератор} последовательно "переключается"{} с одной конструкции на другую до тех пор, пока такие конструкции существуют. Проверка существования следующей конструкции проверяется непосредственно перед переключением. В общем случае конструкций, соответствующих указанному шаблону, может не существовать, в этом случае итерирование происходить не будет (будет 0 итераций).

На каждой итерации в sc-итератор записываются sc-адреса sc-элементов, входящих в соответствующую sc-конструкцию, таким образом найденные элементы могут быть обработаны нужным образом в зависимости от задачи.}
\scnsuperset{трехэлементный sc-итератор}
\scnaddlevel{1}
\scnrelfrom{класс sc-конструкций}{трехэлементная sc-конструкция}
\scnaddlevel{-1}
\scnsuperset{пятиэлементный sc-итератор}
\scnaddlevel{1}
\scnrelfrom{класс sc-конструкций}{пятиэлементная sc-конструкция}
\scnnote{В настоящее время \textit{пятиэлементный sc-итератор} реализуется на основе \textit{трехэлементных sc-итераторов} и в этом смысле не является атомарным. Однако, введение \textit{пятиэлементных sc-итераторов} целесообразно с точки зрения удобства разработчика программ обработки \mbox{sc-конструкций}.}
\scnaddlevel{-1}
