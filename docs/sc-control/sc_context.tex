\scnheader{контекст процесса в рамках программной модели sc-памяти}
\scnidtf{ScContext}
\scnidtf{контекст процесса, выполняемого на уровне программной модели sc-памяти}
\scnidtf{метаописание процесса в sc-памяти, выполняемого на уровне программной модели sc-памяти}
\scnidtf{структура данных, содержащая метаинформацию о процессе, выполняемом в sc-памяти на уровне платформы}
\scnrelto{класс компонентов}{Реализация sc-хранилища}
\scnexplanation{Каждому процессу, выполняемому в sc-памяти на уровне \textit{платформы интерпретации sc-моделей компьютерных систем} (и чаще всего соответствующего некоторому \textit{sc-агенту}, реализованному на уровне платформы) ставится в соответствие \textit{контекст процесса}, который является структурой данных, описывающей метаинформацию о данном процессе. На текущий момент контекст процесса содержит сведения об уровне доступа на чтение и запись для данного процесса (См. \textit{метка уровня доступа sc-элемента}).

При вызове в рамках процесса любых функций (методов), связанных с доступом к хранимым в sc-памяти конструкциям одним из параметров обязательно является \textit{контекст процесса}.}
