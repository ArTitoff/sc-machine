\scnheader{Реализация подсистемы взаимодействия с внешней средой с использованием протоколов на основе формата JSON}
\scnexplanation{В связи с большим числом недостатков протокола SCTP было принято решение о разработке другого протокола на основе какого-либо общепринятого текстового транспортного формата. В качестве такого формата был выбран формат JSON.}
\scnrelto{реализация}{Протокол взаимодействия с sc-памятью на основе JSON}
\scnaddlevel{1}
\scnnote{Данный протокол пока не имеет собственного названия}
\scntext{программная документация}{http://ostis-dev.github.io/sc-machine/http/websocket/}
\scnexplanation{В рамках \textit{Протокола взаимодействия с sc-памятью на основе JSON} каждая команда представляет собой json-объект, в котором указываются идентификатор команда, тип команды и ее аргументы. В свою очередь ответ на команду также представляет собой json-объект, в котором указываются идентификатор команды, ее статус (выполнена успешно/безуспешно) и результаты. Структура аргументов и результатов команды определяется типом команды.}
\scnrelfromlist{достоинство}{\scnfileitem{JSON является общепринятым открытым форматом, для работы с которым существует большое количество библиотек для популярных языков программирования. Это, в свою очередь, упрощает реализацию клиента и сервера для протокола, построенного на базе JSON.};
\scnfileitem{Реализация протокола на базе JSON не накладывает принципиальных ограничений на объем (длину) каждой команды, в отличие от бинарного протокола. Таким образом, появляется возможность использования неатомарных команд, позволяющих, например, за один акт пересылки такой команды по сети создать сразу несколько sc-элементов. Важными примерами таких команд являются  \textit{Команда генерации по произвольному образцу} и \textit{Команда поиска по произвольному образцу}.}}
\scnnote{Можно сказать, что протокол на базе JSON является следующим шагом на пути к созданию мощного и универсального языка запросов, аналогичного языку SQL для реляционных баз данных и предназначенному для работы с sc-памятью. Следующий шагом станет реализация такого протокола на основе одного из стандартов внешнего отображения sc-конструкций, например, \textit{SCs-кода}, что, в свою очередь, позволит передавать в качестве команд целые программы обработки sc-конструкций, например на языке SCP.}
\scnaddlevel{-1}
